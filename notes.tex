\documentclass{article}
\usepackage[english]{babel}

\usepackage[a4paper,top=2cm,bottom=2cm,left=3cm,right=3cm,marginparwidth=1.75cm]{geometry}

% Useful packages
\usepackage{amsmath}
\usepackage{graphicx}
\usepackage[colorlinks=true, allcolors=blue]{hyperref}
\usepackage{biblatex} %Imports biblatex package
\addbibresource{sample.bib} %Import the bibliography file



\begin{document}
\title{MATH420 project}
\author{Hubert Radecki, supervisor: David Schaich}
\maketitle

\begin{abstract}
\end{abstract}

\subsection{notes}
Dark matter terms to investigate further:
\begin{itemize}
\item  supersymmetry / neutralino /WIMP / LSP / R-parity / cold DM
\item  Kaluza-Klein particles , KK parity, randall-sundrum model

\end{itemize}

\subsection {questions}
\begin{itemize}
    \item supplemental materials 
    \item idea nr1: why MACHO's are old fashioned and most of the candidate theories were rulled out. Introduce PBH and how it sheds new light as DM candidate. PBHs microlensing calculations and simulation? compare to observations/ discussion about PBH constraints, contribution to DM 
    \item idea nr2: KK particles and extra dimensions explaining DM phenomenon. Spectrum of KK states. Constraints for KK DM candidate. WIMP + KK DM. scattering of WIMP with nuclei. Comparison of different candidates for KK DM and finding the one that suites the best. how to develop the extra dimension: consider two extra dimensions. Manifolds, tensors, lagrangian, symmetries  
    \item idea nr3: supersymmetry 
\end{itemize}


\section{Introduction/notes}

\begin{enumerate}
    \item Supersymmetry is a principle on which multiple supersymmetric theories base on. Minimal Supersymmetric Standard Model (MSSM) is a model with minimum number of new particles, superpartners of already known particles. 
    \item WIMP
    The lightest supersymmetric particle (LSP) is not decaying (in R-parity model) and therefore may be a candidate for dark matter. It would need to be a particle that only interacts through weak and gravitational interactions and be electrically neutral. The candidates are neutralino (mix of higgsinos and gauginos), gravitino (fermion of spin 3/2), sneutrino (superpartner of a fermion with spin 0). 
    \item Massive Astrophysical Compact Halo Objects (MACHOs)- object that doesn't emit or emits little radiation so it's hard to detect, eg. black holes, neutron stars, dwarfs, objects not composed of baryonic matter, dark energy stars. Ruled out because it's not enough to be a missing mass in the universe associated with dark matter. It also needs to be spread out evenly around galaxies to counter gravity, but these objects are isolated points. 
    \item KK theory - idea of fifth dimension, connected to string theory. Can be presented as small circles (10E-30cm) in every point of universe. 15 components of metric tensor with scalar field called 'dilaton' 

\end{enumerate}


\subsection{KK articles}
\begin{enumerate}
    \item arxiv.org/abs/hep-ph/0701197
    \begin{itemize}
    \item  Two recent propositions: ADD model and UED (Universal extra dimensions) model. ADD model tells the four dimensional fields of a standard model are confined in a brane and gravity is in the bulk. The UED takes a different approach of all dimesional fields propagating universally across extra dimensions.  
    \item UED model assumes a flat manifold with extra dimensions (fermions problem)
    \item in terms of dark matter, extra dimensions might answer the problem by considering it's stable, doesn't emit radiation, is electrically neutral. 
    \item KK parity consequention is that the LKP (Lightest KK particle) is stable. 
    \item KK states - finding which one is the lightest (and stable, non strongly interacting and electrically neutral) to find LKP - DM candidate. Eg KK gauge boson, KK graviton, KK neutrinos, KK photon.
    \item constraints: electroweak precision data, rare decays, flavor physics, magnetic moment of a muon. Theory such as UED must satisfy these constraints.
    \item KK dark matter. As mentioned before dark matter must obey few rules in order to be a possible candidate. That is small mass, no radiation, and the only interactions with SM would appear through weak and gravitational forces based on universe observations. There are many more constraints to be satisfied such as ... . 
    \item KK gravitons- if we consider KK to be the LKP, the process of DM candidate is the same as to superWIMPS.  
    \end{itemize}
    \item https://pdg.lbl.gov/2023/web/viewer.html?file=../reviews/rpp2022-rev-dark-matter.pdf
    \begin{itemize}
    \item properties of DM 
    \item genesis of DM
    \item distribution of DM
    \item astrophysical detection of DM
    \end{itemize}
    \item 
    \begin{itemize}
    \item 
    \end{itemize}
    \item 
    \item 
\end{enumerate}

\subsection{Supersymmetry articles}
\begin{enumerate}
    \item
    \begin{itemize}
    \item  
    \end{itemize}
\end{enumerate}

\subsection{MACHO's - Primordial black holes}
\begin{enumerate}
    \item https://arxiv.org/pdf/1910.01285.pdf
    \begin{itemize}
    \item  Primordial black holes (PBH), formed in the early universe as a result of large primordial density perturbations, are compelling DM candidates if massive enough to survive Hawking evaporation over the age of the universe.
    \item consideration of finite-size source effects leads to the swath of parameter space where PBH can be the totality of the DM being significantly larger than previously thought.
    \item PBHs as DM candidate can be applied to dark quark nuggets, strangelets, axion miniclusters, or axion stars. 
    \end{itemize}
    \item https://arxiv.org/pdf/1701.02151.pdf , 
    \begin{itemize}
    \item explanation for large galaxies in early uniuverse seen by JWST telescope
    \item PBH less than 10^11kg due to hawking radiation evaporation constraint 
    \item xray and background radiation 
    \item https://arxiv.org/pdf/2211.05767.pdf 
    \item https://arxiv.org/pdf/2110.02821.pdf
    \end{itemize}
\end{enumerate}


\subsection{challenges[1]}
\begin{enumerate}
    \item Cuspy halo problem - dark matter distribution around small galaxies. In simulations it should have a cuspy darm matter distribution with density increasing with small radii, but rotation curves of observed galaxies suggest that dark matter is centred. 
    \item Missing satelites problem - number of dwarf galaxies is significantly lower than predictions from amount of dark mater halos equivalents.  
    \item satellite disk problem - dwarf galaxies orbits are not random as predicted from simulations 
    \item high velocity galaxy problem - no bulges at major merges while galaxies were creating 
\end{enumerate}










https://pdg.lbl.gov/2023/web/viewer.html?file=../reviews/rpp2022-rev-dark-matter.pdf








arxiv.org/abs/hep-ph/0701197 - ued
https://arxiv.org/abs/1703.04985
https://arxiv.org/pdf/2306.13408.pdf
https://arxiv.org/pdf/1702.02949.pdf 

New:
https://arxiv.org/pdf/2301.01365.pdf


\end{document}