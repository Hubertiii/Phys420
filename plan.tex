\documentclass{article}
\usepackage[english]{babel}

\usepackage[a4paper,top=2cm,bottom=2cm,left=3cm,right=3cm,marginparwidth=1.75cm]{geometry}

% Useful packages
\usepackage{amsmath}
\usepackage{graphicx}
\usepackage[colorlinks=true, allcolors=blue]{hyperref}
\usepackage[nottoc]{tocbibind}

\begin{document}
\title{MATH420 project}
\author{Hubert Radecki, supervisor: David Schaich}
\maketitle

\begin{abstract}
Dark matter is one of the most discussed subjects in physics. This review summarises 5 of the most promising models and focuses on primordial black holes and their constraints as a dark matter candidate. The possible detection is also discussed.   
\end{abstract}

\tableofcontents

\section{Introduction}
Dark matter in general, possible candidates
Dark matter is an observational fact
Matter in the universe consists of ordinary matter and mysterious dark matter.
Many candidates for dark matter are considered taking into account their properties,...
\section {evidence for Dark matter}
- for spiral galaxies, the mass rotation far from the centre is approximately the same as mass close to the centre which should not be a case using conservation laws (v=sqrtGM/r). So there must be a mysterious mass spread further from the centre of the visible mass that makes these objects rotate at similar speeds.  
-galaxy clusters causing gravitational lensing. Mass of the galaxy can be measured using this technique and the calculations show the visible mass is only a fraction. 
-galaxy collisions can be observed showing the x-ray and visible matter is in a different place than the majority of matter calculated by lensing (bullet clusters)
-simulations of dark matter distribution 
-CMB power spectrum
-JWT guardian article by A.Pontzen
-
\section {DM candidates}
Introduce some candidates and describe how they can be detected, ruled out. 
Look into the mass scale and how some of these candidates are partially or fully ruled out. 
There are many dark matter candidates and the most popular ones are described in the following subsections. 
\subsection{supersymmetry}

Include LSP, R-parity, MSSM, neutralino, gravitino, WIMPs,SIMPs 

\subsection{Kaluza-Klein}
ADD, UED, fifth dimension, KK parity, LKP, KK graviton etc, 

\subsection {axions}

\subsection {neutrinos}

\subsection{MACHO's}
Black holes, PBH, hawking radiation 
A Massive Astronomical Compact Halo Object (MACHO), is an object in space that emits extremely small amount of electromagnetic radiation or no radiation and therefore is hard to detect. 
\section{primordial black holes}
\subsection {black holes}
Two more types of black holes: Anti-de Sitter (AdS) and asymptotically de Sitter (dS) black holes.
De Sitter space has positive cosmological constant, universe has positive curvature, accelerating. 
Anti-de Sitter has negative curvature. We focus on flat universe black holes. 

There are three geometries of black holes in a "no hair" theory, that are the potential descriptions black holes. 
1. Schwarzschild metric is a gemoetric solution for black holes with no angular momentum and no charge. the event horizon is spherical. 
2. Reissner–Nordström metric for static black holes with electric charge. the difference from Schwarzschild black hole is that it has two horizons, event horizon and Cauchy horizon. The limit to the mass of this black hole is when the charge component exceeds the mass component and event horizon disappears. We shall discuss the charge and mass limit for the PBH's in a following sections. 
3. Kerr–Newman metric for rotating black holes with charge. inner horizon, outer horizon and ergosphere 

creation of black holes 

\subsection {PBH}
\subsubsection {formation} How could they be created
1. Inhomogeneities - collapse from primordial density fluctuations. gravitational collapse due to high matter density in inflation and radiation dominated era.
2. scale invariant fluctuations 
3. jeans length and collapse in matter domianted era 
4. more collapse scenarios 

https://arxiv.org/pdf/2002.12778.pdf and https://arxiv.org/pdf/2110.02821.pdf    collapse scenarios 

\subsubsection {present day} how would these look like today? calculations of mass reduction with time due to hawking radiation starting from the start of the universe. 

\subsubsection {detection of PBHs} How can we detect it, JWST telescope explanation for early galaxies

\subsection {constraints/detection}
\subsubsection {maximum and minimum mass/evaporation}
Minimum mass due to Hawking radiation and maximum due to observational constraint.
1. Hawking radiation- black holes are expected to shrink and lose mass and rotational momentum if they do not gain mass. The radiation is inversely proportional to the mass so small black holes evaporate much faster than the big ones. That is a constraint for PBH's for which the PBH's should have minimal mass at the beginning of the universe in order not to disappear by today, so the dark matter ratio is satisfied. 

Evaporation taking into account non-rotating blakc holes with no charge:
1.black holes with mass less than 10e12kg would have evaporated by now - calculations done by Hawking 
2.Page time - evaporation time considering black hole is not a perfect black body, mass less than 4e11kg would have evaporated by now. 
3.These did not take into account new approach to neutrinos where there are 3 flavours of non-zero mass neutrinos. In this case the minimum mass is 5e11kg for black hole to not evaporate by now. 

mention how these minimum mass due to evaporation got more precise with time, eg new neutrino flavour taken into account and search for even more precise modern calculations.

the limit of maximum mass of PBH's is also the survival of galaxies in clusters against tidal disruption by giant cluster PBHs \cite{PBH_as_DM_candidate}. Peculiar velocity of galaxies due to near PBH would give an upper limit to the PBH mass. 

Studying dwarf galaxies can set an upper limit on PBH's by comparing the galaxy structure to an evolution of a structure with theoretical disruptions influenced by PBHs. 


We are assuming the Hawking radiation is valid and it restricts one of the lower mass boundries of PBH's. There also exists a theory questioning Hawking radiation and considering extremely small blackholes as possible candidates for PBH's while the laws of physics for such small and dense object are not known yet and some different rules apply other than Hawking radiation. 

\subsubsection {spin}
\subsubsection {charge}

\subsubsection {microlensing}
brightness changes casued by dark matter moving between observer and the object. \cite{GRIEST_1993_microlensing} 
it must be separated from actual microlensing cases - background such as cosmic rays. 

\subsubsection {gravitational waves}
A rate of gravitational wave events observed by LIGO/VIRGO can set another limit. Massive PBHs would be expected to create a detectable gravitational waves in these observatories.  \cite{PBH_as_DM_candidate}

\subsubsection {cosmic microwave background}
Cosmic microwave background and thermal history of the universe could give another limit for PBH mass. PBHs could absorb a lot of matter in the early stage of the universe boosting the temperature of the background because its increasing luminosity.  \cite{PBH_as_DM_candidate}

\subsubsection{direct detection}
\section {future searches for DM}

\section {conclusion}
PBH vs other DM candidates 









\raggedright
\bibliographystyle{unsrt}
\bibliography{sample}

\end{document}





https://arxiv.org/pdf/hep-ph/9303253.pdf   microlensing 
https://arxiv.org/pdf/2205.02153.pdf       gravitational waves and PBHs
https://arxiv.org/pdf/2002.12778.pdf       PBH creation and mass constraints 
https://arxiv.org/pdf/2206.02672.pdf       PBH constraints with Hawking radiation
